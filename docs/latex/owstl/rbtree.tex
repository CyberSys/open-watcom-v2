% !TeX root = owstl.tex

\chapter{Red-Black Tree}

\section{Introduction}

Class template \code{std::_ow::RedBlackTree} is an implementation of a
red-black tree data structure. It is used as a common base for \code{std::set}
and \code{std::map}. It can be found in \filepath{hdr/watcom/_rbtree.mh}. The
intention was to allow easy replacement and experimentation with other
implementations such as an AVL tree or perhaps some sort of relaxed chromatic
tree suited to concurrent systems.

\section{Status}

The majority of the required functionality has been written. Regression tests
have been written in parallel, but little user testing and feedback exists.

The missing members are:

% Can the list style be modified so there isn't a skip, or maybe a half skip,
% between each item?

\begin{enumerate}
\item \code{reverse_iterator}
\item \code{const_reverse_iterator}
\item \code{template<InputIterator> ctor( InputIterator, InputIterator, ... )}
\item \code{rend( )} and \code{rend( ) const}
\item \code{rbegin( )} and \code{rbegin( ) const}
\item \code{max_size( )}
\item \code{erase( iterator first, iterator last )}
\item \code{swap( RedBlackTree & )}
\item \code{key_comp( )}
\item \code{value_comp( )}
\item \code{find( key_type ) const}
\item \code{count( )}
\item \code{equal_range( key_type )} and \code{equal_range( key_type ) const}
\item Non-member operators and specialized swap algorithm
\end{enumerate}

The completed member are:

\begin{enumerate}
\item \code{iterator}
\item \code{const_iterator}
\item \code{ctor( Compare, Allocator )}
\item \code{cpyctor}
\item \code{operator=}
\item \code{dtor}
\item \code{begin( )} and \code{begin( ) const}
\item \code{end( )} and \code{end( ) const}
\item \code{empty( )}
\item \code{size( )}
\item \code{insert( value_type )}
\item \code{insert( iterator, value_type )} (see N1780)
\item \code{erase( iterator )}
\item \code{erase( key_type cont &)}
\item \code{clear( )}
\item \code{find( key_type )}
\item \code{lower_bound( key_type )} and \code{lower_bound( key_type ) const}
\item \code{upper_bound( key_type )} and \code{upper_bound( key_type ) const}
\item \code{_Sane( )}
\item Internal tree balancing functions
\end{enumerate}

\section{Design Details}

\code{template <class Key, class Compare, class Allocator, class ValueWrapper>}\\
\code{class RedBlackTree}

The type \code{Key} is used to index the tree; the functor \code{Compare}
(class with \code{operator( )} defined) provides ordering to the keys; the
class \code{Allocator} provides the memory allocation; the class
\code{ValueWrapper} defines the type of the objects stored in the tree and
provides an \code{operator( )} that knows how to extract the key from that
type. \code{ValueWrapper} allows the same tree code to apply to sets where the
key is the only thing stored and maps where the object stored has a key and a
mapped value.

\subsection{Relationship to \code{map} and \code{set}}

The templates \code{std::set} and \code{std::map} take their base class as a
template parameter. They select the appropriate value wrapper and inherit all
the functionality. The base currently defaults to \code{RedBlackTree} which is
the only implementation available.

\subsection{Description of a Red-Black Tree}

A red-black tree is an ordered binary tree made up of nodes, where each node
can have up to two children. An ordered binary tree orders the nodes so that a
left child is less than its parent and a right child is greater. It could be
the other way around, and this implementation uses a comparison function and
puts the child on the left if \code{compare( child, parent )} evaluates true.
If a node has no children it is a leaf, otherwise it is an internal node. Some
implementations only hold the actual data in the leaves and the internal nodes
are just placeholders. This implementation has imaginary leaves - null
pointers. If a node's child pointer is null then that non-existent child is a
leaf, and we hold all the data in the real, existing nodes. Therefore, there
is no special leaf node type, just a null pointer if there is no child with
data.

A red-black tree adds a color to every node, and defines some rules that mean
the tree stays balanced. A tree is balanced if the difference between the
largest and smallest depth of a leaf is bounded. The invariants are:

\begin{enumerate}
\item Every red node has a black parent/
\item Every route from the root node to a node with 0 or 1 children has the
same number of black nodes.
\item Every leaf is black (note this is assumed as leaves don't really exist
in this implementation).
\item The root is black.
\end{enumerate}


This data structure has been well covered in the literature, for a more
detailed information see: (Prof Lyn Turba, Wellesly College, CS231 Algorithms
Handout 21, 2001) (McGill University, Notes for 308-251B,
http://www.cs.mcgill.ca/~cs251/, 1997) \todo{Look into setting up a BibTeX
database for proper references and citations.}

\subsection{Overview of the class}

The tree class defines an internal \code{Node} structure that is made up of
the object stored in the tree, node pointers for the parent and left and right
children, and the node color. There is an allocator member object,
\code{mMem}, that is rebound to allocate \code{Node} objects. There are
pointers to the root and furthest left and right nodes. These are used to mark
were to start the search, and create the \code{begin} and \code{end} iterators
respectively. The \code{iterator} and \code{const_iterator} classes are
derived from a common member class. There is an \OW\ extension method
\code{_Sane( )} that checks the integrity of the data structure. Related to
this is an integer \code{mError} member that is assigned a value if an error
is detected when \code{_Sane} is run. \todo{This should perhaps be renamed
\code{_Error} or made private and a \code{_GetError( )} method provided.}

\subsection{Inserting Elements and Rebalancing}

The \code{insert} method calls \code{unbalancedInsert} and
\code{insBalTransform}. The loop in \code{unBalancedInsert} moves from child
to child searching for the leaf of the tree where the new item can be
inserted, similarly as the \code{find} algorithm checks for the item. A final
check is made at the end of the loop to see if the key already exists. If it
does, an iterator to the existing key is returned. Otherwise, a new node is
allocated and constructed. The node is linked into the tree at the place
found. A try-catch is placed around the construction of the node to deallocate
the node again if any exceptions are raised. This is needed to stop a memory
leak that could occur because the memory has been allocated, but the exception
has stopped the node from being linked into the tree (so it would never get
destroyed when the tree is destroyed).

At this point, the tree is a valid binary tree but does not necessarily obey
the red-black balance criteria. The new node is painted red so as not to
invalidate the black-height rule, but this may introduce a violation of the
red-red rule. Function \code{insBalTransform} is called with a pointer to the
newly inserted node to correct this. This is where this implementation of a
red-black tree varies from the most common implementations. Usually the
balancing procedure is broken down into a series of rotations where a subtree
of the tree would appear to be rotated if represented graphically. These
rotations can be left or right and the procedure moves up to the parent
subtree and is repeated until the violation is removed.

Instead, \OW\ uses the concept of a transformation. (Alternatives to Two
Classic Data Structures, Chris Okaski, 2005?) A sub-branch of the tree is
analyzed to see which case it matches and the elements in that branch are then
reorganized and recolored in one block of code. Although this isn't wildly
different, it was hoped that it would allow a faster algorithm to be created
because larger subtrees and special cases could be matched and manipulated in
one go, and the code generator may be able to make a better job of optimizing
the code because a larger block of manipulating instructions would be
together. Whether this was a good decision will be born out in time.

\todo{Explain why the insert methods are currently inline - compiler bug -
what exactly was the problem?}

\subsection{Deleting Elements}

Deletion is a bit more complicated than insertion. The main method that gets
called is \code{erase( iterator )}.
I did hope it may be possible to rewrite this in a way that is easier to
understand. There are two main cases:

\begin{itemize}
\item The node to be removed has both children.
\item The node has one or more null children (i.e. has 0 or 1 real child, in
other words 1 or 2 leaves) - I've called this an ``end node.''
\end{itemize}

If it is an end node (has 0 or 1 real child) then that child can be linked
into its place or the node can just be deleted. We take note of the deleted
node's parent, the child, and its color. The other case where it has 2
children is more complicated. We swap the predecessor (which cannot have a
right child by definition) of the deleted node into the place of the deleted
node, and change its color so that part of the tree is still valid. The node
being removed is now effectively the predecessors old position, so we take
note of its original child, parent and color.

Now we have created a situation where we are really removing an end node. We
can look at the color of the node to be removed. If it is red then there is no
violation of the black height rule by removing it. Also, it cannot have a real
child, so there is nothing to link in its place. If the removed end node is
black there are two cases. If it has a child, that child must be red or there
would have been a black height violation, thus we link the child in the place
and paint it black to maintain the black height. If there was no child, we
have created a black height problem - there is a lack of black on this branch.
The deleted node has left a null leaf node in its place, we usually count
these as black, but in this case we have to call it double black to resolve
the black height problem. This isn't valid so we call
\code{doubleBlackTransform( )} to run through a set of cases to rearrange
subtrees and remove the need for double black.
